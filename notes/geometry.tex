%!TEX root = cv_ml_notes.tex
\section{Geometry Processing} 

\subsection{Basics}

Plane normal equation

Plane point distances

Barycentric coordinates

\subsection{ Surface Representations}

Explicit:
\begin{itemize}
\item Mesh
\item Spline surface
\item Oriented planes
\item Point cloud
\end{itemize}

Implicit - voxel grid:
\begin{itemize}
\item Signed distance fields (implicit) <0, 0, >0
\item Signed distance fields (implicit)
\item Occupancy grid
\item Signed-distance grid
\item Voxel octree
\item Tetrahedral Mesh
\end{itemize}

Volumetric modeling for vision:
\begin{itemize}
\item Flexible and robust surface representation
\item Handles (changes of) complex surface topologies effortlessly
\item Ensures watertight surface / manifold / no self- intersections
\item Allows to sample the entire volume of interest by storing information about space opacity
\item Voxel processing is often easily parallelizable
\end{itemize}

Drawbacks: Implicit surface representation

\subsection{Marching Cubes}
\begin{itemize}
\item Recovers an isosurface from a volume
\item ensures a watertight surface
\item Can be done per voxel
\item 15 combinations of surface intersections per cube
\item Precise normal specification
\item Accuracy depends on resolution
\end{itemize}

Trivial merging or overlapping of different surfaces based on the corresponding implicit functions:
minimum of the values for merging, averaging for overlapping

Limitations of Marching Cubes
\begin{itemize}
\item  Maintains 3D entries rather than a 2D surface, i.e., higher computational and memory requirements
\item  Generates consistent topology, but not always the topology you wanted
\item  Problems with very thin surfaces if resolution not high enough
\end{itemize}

\subsection{ICP}

Point to point

Point to plane

\subsection{Point Cloud Merge}

\begin{enumerate}
\item ICP for point cloud matching
\item Normal Estimation
\item Outlier detection / removal 
\item Surface / mesh fitting / template fitting
\end{enumerate}

\subsection{Geometric Representations}

Bezier curves

\subsection{Laplacian Deformation}