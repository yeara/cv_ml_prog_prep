\refstepcounter{chapter}
\chapter{Geometry Processing} 

\section{Basics}

Plane normal equation

Plane point distances

Barycentric coordinates


\section{ Surface Representations}

Explicit:
\begin{itemize}
\item Mesh
\item Spline surface
\item Oriented planes
\item Point cloud
\end{itemize}

Implicit - voxel grid:
\begin{itemize}
\item Signed distance fields (implicit) <0, 0, >0
\item Signed distance fields (implicit)
\item Occupancy grid
\item Signed-distance grid
\item Voxel octree
\item Tetrahedral Mesh
\end{itemize}

Volumetric modeling for vision:
• Flexible and robust surface representation
• Handles (changes of) complex surface topologies effortlessly
• Ensures watertight surface / manifold / no self- intersections
• Allows to sample the entire volume of interest by storing information about space opacity
• Voxel processing is often easily parallelizable


Drawbacks:
Implicit surface representation

\subsection{Marching Cubes}
Recovers an isosurface from a volume
ensures a watertight surface
Can be done per voxel
15 combinations of surface intersections per cube
Precise normal specification
Accuracy depends on resolution

Trivial merging or overlapping of different surfaces based on the corresponding implicit functions:
• minimum of the values for merging • averaging for overlapping

Limitations of Marching Cubes
• Maintains 3D entries rather than a 2D surface, i.e.,
higher computational and memory requirements
• Generates consistent topology, but not always the topology you wanted
• Problems with very thin surfaces if resolution not high enough


\section{ICP}

Algorithm:

\section{Point Cloud Merge}

ICP for point cloud matching

Normal Estimation

Outlier detection / removal 

Surface / mesh fitting / template fitting

\subsubsection{Geometric Representations}

Bezier curves

\section{Laplacian Deformation}